\section*{Abstract}
The new MeerKAT Radio Interferometer poses an image reconstruction problem on a large scale. It measures an incomplete set of Fourier components, which have to be reconstructed by an imaging algorithm.  Compressed Sensing reconstructions have the potential to improve the effective accuracy of MeerKAT, but so far have higher runtime costs compared to state-of-the-art CLEAN implementation. Both Compressed Sensing and CLEAN reconstructions use the non-uniform FFT approximation to cycle between Fourier and image space. But compared to CLEAN, Compressed Sensing algorithms need more cycles to converge, which is one of the reasons why they have higher runtime costs.

In this project, we investigate if Compressed Sensing algorithms can reduce the costs by replacing the non-uniform FFT approximation. We discuss three alternatives and create a new algorithm which does not need the non-uniform FFT during optimization. We leveraged the starlet transform and created a Compressed Sensing algorithm, which only needs to calculate the transform for non-zero basis functions. This allows us to use the direct Fourier Transform, which naturally extends to wide field of view measurements without any approximations. We demonstrate superior image quality on simulated data and extrapolate the runtime costs to a real-world MeerKAT observation. Sadly, our algorithm was unable to reduce the runtime costs of large scale reconstructions.



Currently, the non-uniform FFT seems an efficient way to approximate the Fourier Transform, which leads to iterative reconstruction algorithms. The number of measurements continue to rise in near future. At some point, distributed computing will become necessary for large scale reconstructions. The iterative nature of the non-uniform FFT becomes an issue for distributed computing. It is still an open question how to create a distributed large scale reconstruction algorithm. As it is, our approach is too expensive. But with future approximations, we may become competitive.
