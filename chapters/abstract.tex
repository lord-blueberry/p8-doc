\section*{Abstract}
The new MeerKAT Radio Interferometer poses an image reconstruction problem on a large scale. It measures an incomplete set of Fourier components, which have to be reconstructed by an imaging algorithm.  Compressed Sensing reconstructions have the potential to improve the effective accuracy of MeerKAT, but so far have higher runtime costs compared to state-of-the-art CLEAN implementation. Both Compressed Sensing and CLEAN reconstructions use the non-uniform FFT approximation to cycle between Fourier and image space. But compared to CLEAN, Compressed Sensing algorithms need more cycles to converge, which is one of the reasons why they have higher runtime costs.

In this project, we investigate if replacing the non-uniform FFT approximation reduces the runtime costs of Compressed Sensing reconstructions. We discuss three alternatives and decided to create a new algorithm using the direct Fourier Transform. We leveraged the starlet transform and created a Compressed Sensing algorithm, which only needs to calculate the transform for non-zero basis functions. Our algorithm does not need iterative approximation algorithms of the Fourier Transform. With Coordinate Descent as the optimization algorithm, our approach lends itself to distributed image reconstruction. Although our algorithm was able to reduce the runtime costs it is still too expensive for large scale reconstructions.

Currently, there is no clear alternative to the non-uniform FFT approximation. Creating a cost effective and distributable Compressed Sensing algorithm is still an open problem.