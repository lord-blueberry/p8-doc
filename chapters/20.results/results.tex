\section{Test on Simulated Data}\label{results}
In this section, we test our new approach with Coordinate Descent on simulated MeerKAT data. We show that our approach does not need to calculate the whole Fourier Transform matrix $F$. Instead, we use a heuristic to only use relevant columns. As mentioned in section \ref{cd}, it is unknown if our approach will converge to the true optimum. Nevertheless, we compare our results with CASA's CLEAN implementation, and demonstrate super-resolution performance of Coordinate descent together with accurate total flux modelling.

The two simulated datasets contains idealized MeerKAT observations. Compared to the real world, the two simulated datasets contain few Visibilities and not representative of the real data volume. Also, more realistic simulations which contain pointing-, calibration-, and thermal noise are out of scope for this project. The simulations are used to isolate the two fundamental issues in radio interferometer image reconstruction: Non-uniform sampling and incomplete measurements.

\subsection{Super-resolution of two point sources}
The first simulated observation contains two non-zero pixels, i.e. point sources, with intensity of 2.5 and 1.4 Jansky/Beam. The image has a size of $256^2$ at a resolution of 0.5 arc-seconds per pixel. The integral, the total Flux of the image, is 3.9 Jansky/beam.

\begin{figure}[h]
	\centering
	\begin{subfigure}[b]{0.4\linewidth}
		\includegraphics[width=\linewidth, trim={0.2in, 0.2in, 0, 0.2in}, clip]{./chapters/20.results/points/tclean_points.png}
		\caption{CLEAN reconstruction \\with CASA standard parameters.}
		\label{results:points:tclean}
	\end{subfigure}
	\begin{subfigure}[b]{0.4\linewidth}
		\includegraphics[width=\linewidth, trim={0.2in, 0.2in, 0, 0.2in}, clip]{./chapters/20.results/points/cd_points.png}
		\caption{Coordinate Descent reconstruction\\ with $\lambda = 0.01, J=4$.}
		\label{results:points:cd}
	\end{subfigure}
	
	\caption{Image reconstruction of two simulated point sources.}
	\label{results:points}
\end{figure}

The figure \ref{results:points} shows the CLEAN and the Coordinate Descent reconstruction. CLEAN reconstructs the image \ref{results:points:tclean} at the accuracy limit of the instrument. It essentially reconstructs a blurred version of the observed image, where the blurring represents the accuracy of the instrument. With compressed sensing, we aim to reconstruct the de-blurred image, increasing the effective accuracy of the instrument. 

Coordinate Descent in image \ref{results:points:cd} shows a super-resolved reconstruction of the two point sources. It reconstructs two narrow peaks surrounded,by a low-intensity Gaussian emission. Also, Coordinate Descent manages to capture the total flux more accurately than CLEAN.  The total flux of image \ref{results:points:cd} results in 3.92 Jansky/beam, compared to CLEAN which overshoots the number by a factor of a 1400. The total flux of the CLEAN reconstruction \ref{results:points:tclean} overshoots the 3.9 figure by a factor of 1400. This gets obvious when we compare the intensity profile of CLEAN, Coordinate Descent and the ground truth in figure \ref{results:points:contour}.

\begin{figure}[h]
	\centering
	\includegraphics[width=0.8\linewidth]{./chapters/20.results/points/contour_points.png}
	\caption{Intensity profile of the two point sources.}
	\label{results:points:contour}
\end{figure}

CLEAN essentially places a Gaussian function with correct peak intensity at the point source location, but it does not respect the total flux of the image. Coordinate Descent keeps the total flux in mind. The pixels in each area of the point sources sum up to the correct values of 2.5 and 1.4 respectively. However, note that Coordinate Descent in figure \ref{results:points:contour} seems to have both point sources shifted by approximately a pixel. It looks suspiciously like an off-by one error. Sadly in the time frame of this project, no error was found or an explanation for this behaviour.

%Possible 

\subsection{Super resolution of mixed sources}

Dataset of three gaussian extended emissions and sixteen point sources at varying intensities.

Parameters of Coordinate Descent

super resolution of CD. However, it did not find all point sources
Lets look closer at the flux reconstruction of extended emissions

\begin{figure}[h]
	\centering
	\begin{subfigure}[b]{0.4\linewidth}
		\includegraphics[width=\linewidth, trim={0.2in, 0.2in, 0, 0.2in}, clip]{./chapters/20.results/mixed/mixed_clean.png}
		\caption{CLEAN reconstruction}
		\label{results:mixed:tclean}
	\end{subfigure}
	\begin{subfigure}[b]{0.4\linewidth}
		\includegraphics[width=\linewidth, trim={0.2in, 0.2in, 0, 0.2in}, clip]{./chapters/20.results/mixed/mixed_cd.png}
		\caption{Coordinate Descent Reconstruction}
		\label{results:mixed:cd}
	\end{subfigure}
	\caption{Reconstruction on mixed sources}
	\label{results:mixed}
\end{figure}

Question of Flux reconstruction

 $\lambda$ for different starlet layers like in \cite{girard2015sparse}

\begin{figure}[h]
	\centering
	\begin{subfigure}[b]{0.3\linewidth}
		\includegraphics[width=\linewidth, trim={0.2in, 0.8in, 3.2in, 1.8in}, clip]{./chapters/20.results/mixed/mixed_cut_model_line.png}
		\caption{Ground truth.}
		\label{results:mixed:cut0:img}
	\end{subfigure}
	\begin{subfigure}[b]{0.6\linewidth}
		\includegraphics[width=\linewidth, trim={0, 0, 1in, 0.2in}, clip]{./chapters/20.results/mixed/mixed_cut0.png}
		\caption{Intensity profile.}
		\label{results:mixed:cut0:profile}
	\end{subfigure}
	\caption{Intensity profile.}
	\label{results:mixed:contour}
\end{figure}

\begin{figure}[h]
	\centering
	\begin{subfigure}[b]{0.3\linewidth}
		\includegraphics[width=\linewidth, trim={0.2in, 0.2in, 3.2in, 0.2in}, clip]{./chapters/20.results/mixed/mixed_cut_model2_line.png}
		\caption{ground truth}
		\label{results:mixed:cut0}
	\end{subfigure}
	\begin{subfigure}[b]{0.6\linewidth}
		\includegraphics[width=\linewidth, trim={0.2in, 0.2in, 0, 0.2in}, clip]{./chapters/20.results/mixed/mixed_cut2.png}
		\caption{ground truth}
		\label{results:mixed:cut0}
	\end{subfigure}
	\caption{Reconstruction on mixed sources}
	\label{results:mixed:contour}
\end{figure}


Coordinate Descent did not reconstruct all point sources. How many starlets are non-zero is the major point for runtime. It depends on how many areas of the image are non-zero. Starlet has a representation for extended emission, how many starlets are needed for modelling is hard.

Runtime problems
244 non-zero starlets, 

show the areas


