In this section, we compare the Coordinate Descent reconstruction with CLEAN on simulated MeerKAT data, and compare the runtime complexity of the two approaches on a real-world MeerKAT observation. We show that Coordinate Descent reconstructs the image by only computing a subset of the Fourier Transform Matrix' columns $F^{-1}$, and investigate if this approach reduces the runtime complexity on the large scale reconstruction problems of MeerKAT.


The real world MeerKAT data were calibrated and averaged down to reduce its size to 88 Gigabytes. The raw, uncalibrated data ranges between 500 and 1000 Gigabytes. Data on this scale requires a mature pipeline for reading and image reconstruction. Within the time limit of this project, only a reconstruction with the WSCLEAN\cite{offringa2014wsclean} pipeline was possible.

The simulations were created with the Meqtrees software package. Two simulations which contain roughly (size of Visibilities) perfectly calibrated Visibilities were created. 

\subsection{Imaging on Simulated Data}
We compare the reconstruction of Coordinate Descent and CLEAN on two simulated observations. The first observation contains two point sources, on which we show that Coordinate Descent is able to locate the sources below the accuracy limit of the instrument. The second observation contains point and Gaussian sources, on which we show that Coordinate Descent better captures the intensity profile of extended emissions. Also, we look at the shortcomings of the proof-of-concept implementations in terms of reconstruction quality and runtime. We use the CLEAN implementation of CASA in this section. CASA is an established software framework for radio interferometer image reconstruction and was already used in a previous project.


We used CASA's default parameters for CLEAN, except for the maximum number of CLEAN iterations, which we set to 250. The proof-of-concept Coordinate Descent implementation has three parameters to tune: Number of iterations, the number of starlet layers $J$, and the regularization parameter $\lambda$. The first two parameter could be estimated by the reconstruction algorithm itself. In this implementation, it was left to the user. For the two simulations the Coordinate Descent parameters were chosen:
\begin{itemize}
	\item Two Point sources: 4 full iterations, $J=3$, $\lambda=0.1$
	\item Mixed sources: 4 full iterations, $J=7$, $\lambda=0.1$
\end{itemize}

In figure \ref{results:point} shows an image of the Coordinate Descent and CLEAN reconstruction. Figure \ref{results:points:contour} shows the intensity profile of both reconstructions compared with the ground truth. Note that figure \ref{results:points:contour} has a logarithmic y axis. The reconstructions differ in two notable ways: The peak intensity of the source, and how much each algorithm spreads the point source. 

\textit{Spread}: CLEAN reconstructs a convolved version of the true image. CLEAN locates the two peaks in figure \ref{results:points:contour}, but convolves the point source with a Gaussian function. CLEAN reconstructs two Gaussian functions with same peak as the point source. The Gaussian represents the accuracy of the instrument. With compressed sensing, we try to reconstruct the observed image above the accuracy of the instrument. In the intensity profile, Coordinate Descent reconstructs two narrow gaussian-like peaks, essentially super-resolving the image. However, note that the total intensity is a fraction of the original peaks.

\textit{Intensity:} [Total Flux] The reconstructions differ in intensity. The figure \ref{results:points:contour} shows the intensity pr. Also note that Coordinate Descent has shifted the smaller peak by a pixel in its reconstruction. [Total flux correct in CD?]



The difference in intensity becomes more apparent with extended emissions. Figure \ref{results:mixed} shows Coordinate Descent and tCLEAN on the simulation with mixed sources. Again the two reconstructions arrive at different intensities. The gaussian emissions are reconstructed with a higher intensity than Coordinate Descent.






With starlets, coordinate descent has a representation for extended emissions. Looking at the intensity profile of an extended emisions \ref{}, we see Coordiante Descent coming closer to the true intensity. Although in the four iterations,  [it still had a considerable margin on error]. 



