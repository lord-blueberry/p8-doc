\section{Eliminating the Major Cycle}\label{killmajor}
Although Compressed Sensing algorithms produce higher quality reconstructions, the CLEAN algorithm so far has lower runtime costs. Both use a version of the Major Cycle architecture for efficient transformation of Visibilities to image space, and to correct the effects of wide field of view imaging. This project explores different architectures for Compressed Sensing, with the goal to decrease the runtime costs in the context of MeerKAT.

Many Compressed Sensing reconstructions use the Major Cycle architecture\cite{girard2015sparse, dabbech2018cygnus, mcewen2011compressed, pratley2018fast}  and do not discuss alternatives. Runtime improvements are explored in the context of the Major Cycle. Pratley et al\cite{pratley2017robust} optimized the non-uniform FFT in the context of Compressed Sensing reconstructions. Dabbech et al.\cite{dabbech2017wEffect} investigated $w$-term approximations for reducing runtime costs of Compressed Sensing reconstructions. 

An alternative to the Major Cycle was investigated by Hardy et al.\cite{hardy2013direct}. They replaced the non-uniform FFT with the direct Fourier transform and distributed the problem.

We

We want explore further and discuss. why they don't work.
In this section, we discuss three possible alternatives to the Major Cycle: Separating the Major Cycle in two Optimization problems, using Spherical Harmonics,

Chosen to do direct fourier transform.



\subsection{Separating the Major Cycle into two Optimization Problems}
In each iteration, the Major Cycle reduces the error from both, the non-uniform FFT approximation, and from incomplete measurements. Here, we discuss the potentials and problems of separating the problem into two optimization objectives \eqref{killmajor:sep:major} \eqref{killmajor:sep:deconv} and solve each after the other. This architecture only needs the dirty image for further processing, which is several times smaller than the original Visibility measurements.

\begin{alignat}{1}
\underset{I_{dirty}}{minimize} \:& \left \|  V - \tilde{F}I_{dirty} \right \|_2^2\label{killmajor:sep:major}\\
\underset{x}{minimize} \:& \left \| I_{dirty} - x \star PSF \right \|_2^2 + \lambda \left \| P(x) \right \|_1 \label{killmajor:sep:deconv}
\end{alignat}

The first objective \eqref{killmajor:sep:major} uses the non-uniform FFT approximation $\tilde{F}$ and searches for the optimal dirty image matching the Visibilities. The optimization algorithm for the first objective looks similar to the Major Cycle architecture. It still uses the non-uniform FFT approximation, and iteratively transforms the Visibilities to image space and back. In this architecture, we can use specialized optimization algorithms potentially uses fewer non-uniform FFT approximations to converge to the dirty image. At this point, we can drop the original Visibility measurements and only use the dirty image for further processing. The error introduced by the incomplete set of Visibilities can be represented as a convolution in image space with a Point Spread Function. The second objective \eqref{killmajor:sep:deconv} therefore can minimized to reconstruct the observed image $x$, which just needs the dirty image and the $PSF$.

The downside of this architecture lies in the $PSF$ of the second objective. The $PSF$ varies over the image. For the most accurate result, we would need a $PSF$ for each pixel. The naive way to calculate a $PSF$ for a pixel is to set all amplitudes of our Visibilities to 1, shift the phase to the desired pixel, and calculate the Fourier Transform. The naive way to calculate a $PSF$ for each pixel leads to a quadratic runtime.

CLEAN gets away with a constant $PSF$\footnote{CLEAN implementations in Radio Astronomy typically use a constant $PSF$. This is not necessary, one can also implement CLEAN with a varying $PSF$.}, because of the Major Cycle architecture. In each cycle, it can reduce the error it introduced in the previous cycle, leading to a more accurate reconstruction. In the new architecture, we need several $PSF$s for a comparable accuracy to the Major Cycle, and this number dictates whether we can reduce the runtime costs. The question, how many $PSF$s are necessary, seems currently unknown in the Radio Astronomy community, as it becomes irrelevant for the Major Cycle. As of the time of writing, we do not know of any published research into this area. We do not have good estimates on the viability of this architecture.

The main advantage for this architecture is that it reduces the problem to a uniformly-sampled grid, and ignore the original Visibilities in later steps. However, in the context of self-calibration, we cannot ignore the non-uniformly sampled Visibilities. We need a transformation from the image back to the original Visibilities and improve the calibration parameters. Self-calibration negates the main advantage of the architecture. Unless we find a way to map the Visibilities together with their calibration parameters on a uniformly-sampled grid, this architecture does not seem practical for MeerKAT.


\subsection{Spherical Harmonics}
Spherical Harmonics are a different way to represent the measurements of an interferometer. Carozzi\cite{carozzi2015imaging} derived a measurement equation based on the fact that the Visibilities fulfils the Helmholtz equation. In practical terms, Carozzi showed we can replace the non-uniform FFT with the Spherical Harmonics Transform in the Major Cycle architecture. Although the Spherical Harmonics Transform has interesting properties for full-sky imaging, this change alone neither leads to a different architecture, nor to a runtime improvement\footnote{The fast Spherical Harmonics Transform is a generalization of the non-uniform FFT\cite{kunisnonequispaced} and is still an active field of research\cite{schaeffer2013efficient}.}

Compared to the Fourier domain, Spherical Harmonics naturally represent wide field of view interferometers and do not have a third dimension. This opens up new designs for Compressed Sensing reconstructions. For example, we can reconstruct the image by in-painting the missing Spherical Harmonics. Because they naturally represent curved surfaces, we can in-paint in a two dimensional space instead of three. Another direction is to image on the sphere directly. MCewen et al.\cite{mcewen2008simulating} showed a way to put two dimensional wavelets on the sphere. A Compressed Sensing reconstruction can use a spherical wavelet as regularization and may simplify the transformation between measurements and reconstruction space.

Sadly, there is little published research in this area for Radio Astronomy image reconstructions. MCewen et al.\cite{mcewen2008simulating} improved the runtime costs of simulations with the spherical Haar wavelet. Later work\cite{mcewen2011compressed} showed a proof-of-concept Compressed Sensing reconstruction projected on the sphere, which improved the image quality. However, it did not use Spherical Harmonics and \cite{mcewen2011compressed} is still based on the Fourier transform. At this point we have not found a proof-of-concept reconstruction algorithm which uses Spherical Harmonics to reduce runtime costs. 

Although there seems untapped potential for new, cheaper reconstruction algorithms with Spherical Harmonics, practical limitations may explain the lack of research in this area. The Fourier relationship, Calibration, the effect of the ionosphere, everything discussed in section \ref{meerkat} is based on a plane wave arriving at the instrument\cite{thompson1986interferometry, smirnov2011revisiting}. Spherical Harmonics are not. They are derived from a different property of the signal. The plane wave formalism may not translate to Spherical Harmonics. We may need to re-invent self-calibration, ionosphere distortion and more for Spherical Harmonic reconstructions.


\subsection{Direct Fourier Transform on a large scale}
The direct Fourier Transform does not use the non-uniform FFT approximation. It uses the full $F$ matrix in \eqref{intro:cs}. It naturally extends to wide field of view imaging, and can handle the $w$-term without any approximation algorithms. Using the direct Fourier Transform simplifies the architecture of the reconstruction. Without the approximation algorithm. The drawback is either a large memory requirement for caching, or high runtime costs for continually calculating the transform matrix. For MeerKAT reconstructions where we reconstruct millions of pixels from several billions of Visibilities, the matrix becomes too large for any practical application.

However, the full Fourier matrix is not needed.

Sparse in the image domain. Only Columns necessary of non-zero pixels.

Even more sparse in a different domain, for example Starlets \cite{girard2015sparse}. or Daubechies Frames\cite{dabbech2018cygnus}.

If we can estimate which columns are likely non-zero, we just need to calculate a subset of the Fourier matrix $F$.