\section{Compressed Sensing reconstructions in the MeerKAT era}
MeerKAT poses a large scale reconstruction problem. Compressed Sensing approaches exist, they use the major cycle, but they all are slower to compute than CLEAN, for which the major cycle was developed. 

It uses the non-uniform FFT to approximate the Fourier transformation of non-uniformly sampled Visibilities.

This project set out to explore different architectures for Compressed Sensing Reconstructions. A different way of getting from Visibilities to image. 

Three different possibilities, with calculating the non-uniform FFT once, using spherical harmonics, or calculating only columns of the Fourier transform matrix

For the former, we created a novel approach which does not need the major cycle. We need a transformation, where we can both estimate which components are likely to be non-zero, and which use few non-zero components to represent an image.

This project used starlets, which were developed for astronomical images. Their solution is not sparse enough to scale to MeerKAT problems.

This approach does not scale. Compared to major cycle algorithms, we need a space where the image can be VERY sparsely represented, and even 

Different approaches have


Runtime is a major drawback of Compressed Sensing reconstructions.

This project searched if the architecture can be optimized for Compressed sensing.

 How to get the runtime down at least to comparable levels of CLEAN is still an open question.
Still an open question








