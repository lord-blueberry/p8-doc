\section{Compressed Sensing Reconstructions in the MeerKAT era}
Compressed Sensing and CLEAN based algorithms both use the non-uniform FFT to cycle between Visibility and image space. In this environment, Compressed Sensing needs more cycles to converge than CLEAN. This leads to iterative Compressed Sensing algorithms, which becomes difficult to distribute for the large scale MeerKAT problems. We postulated that one may reduce the runtime cost of Compressed Sensing reconstructions by replacing the non-uniform FFT approximation. Three alternatives were discussed: Optimal Projection on a uniform Grid, using Spherical Wave Harmonics and the direct Fourier Transform.

The optimal projection on a uniform grid is way of reducing the problem to a more manageable size. The grid size is usually a fraction of the original non-uniform measurements for MeerKAT reconstructions. The first step is to search for the optimal projection of Visibilities on a uniform grid. We then reconstruct the image with uniformly sampled Visibilities and do not need the original measurements. However, self-calibration requires a transformation from uniform sampled grid back to the original measurements. Unless one finds a way to project the calibration problem on a uniform grid too, self-calibration destroys the main advantage of this approach.

The second alternative is using Spherical Wave Harmonics instead of the Fourier Transform. It is a new way to represent the Radio Interferometric Measurement Equation and has only recently gained attention again. We analysed published research in this area\cite{carozzi2015imaging, mcewen2008simulating} and see potential for a new reconstruction architecture, which may reduce the overall runtime costs. At this point, it is unclear if or how many imaging problems have to be re-investigated by moving to Spherical Wave Harmonics. The Fourier Transform, self-calibration and the current formalism for Radio Interferometric Measurement equation is based on the plane wave\cite{smirnov2011revisiting}. Spherical Wave Harmonics are not. A real-world reconstruction algorithm with Spherical Wave Harmonics may have to deal with imaging problems in a fundamentally different way.

The last alternative, the direct Fourier Transform, uses the explicit matrix instead of an approximation algorithm. Previous attempts\cite{hardy2013direct} lead to an impractically large matrix for MeerKAT reconstructions, but showed potential for distributed computing. In this project, we improved the runtime costs and memory requirement of the direct Fourier Transform. We used Coordinate Descent together with the starlet transform and created an algorithm which only calculates the direct Fourier Transform for non-zero basis functions. Sadly, our improvements alone were not enough to scale the direct Fourier Transform to MeerKATs data volume. Compared to the non-uniform FFT based algorithms, our approach leads to higher runtime costs for the same measurements. 

In this project we have not found a clear alternative to the non-uniform FFT. It is readily available and currently seems to be the most efficient approximation of the Fourier Transform. The $w$-stacking algorithm\cite{offringa2014wsclean} has introduced some level of distribution for the non-uniform FFT operation, and state-of-the-art Compressed Sensing algorithms\cite{dabbech2018cygnus, pratley2018fast} are developed with distributed computing in mind. Still, the non-uniform FFT tends to dominate the runtime for large scale reconstructions. How to effectively distribute a reconstruction algorithm based on the non-uniform FFT is still an open problem.

With an expansion of the MeerKAT instrument already planned, the problem sizes will only increase in the near future. This may force reconstruction algorithms to increasingly rely on approximations instead of exact operations. Indeed, the path to a scalable, distributable reconstruction algorithm may lie in specialized approximations for MeerKAT. The large number of Visibilities are bound to contain redundant information. By increasing the data volume, one also increases the amount of redundant information. A given reconstruction algorithm may waste resources on measurements, which contribute little to the final reconstruction.

To our knowledge, it is unknown how much redundant information is measured by Radio Interferometers, and how this affects the runtime costs of reconstruction algorithms. The non-uniform FFT is a general purpose approximation and does not account for redundant information and the runtime costs are unlikely to change significantly by removing the redundant information from the Visibilities. The direct Fourier Transform on the other hand may benefit significantly, and might result in an efficient alternative to the non-uniform FFT when we account for the redundant information inherent in the measurements.

With the theory of Compressed Sensing, we can design algorithms that leverage the redundant information. To our knowledge, the Radio Astronomy community has not yet investigated the potential benefits of accounting for redundant information. State-of-the-art Compressed Sensing algorithms use the $w$-stacking algorithm and include all redundant information in the reconstruction.

Accounting for redundant information may be a way to reduce the problem size of future MeerKAT observations. But as we have seen in this project, single change in isolation likely leads to higher runtime costs. In the case of the direct Fourier Transform, it needs further approximations on top of the improvements developed in this project before it may even be tractable for MeerKAT reconstructions. Accounting for redundant information alone is unlikely to improve the runtime costs of Compressed Sensing reconstructions. 

The hope is that it does lead to a future 















