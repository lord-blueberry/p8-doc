\section{Compressed Sensing reconstructions in the MeerKAT era}
Current Compressed Sensing reconstructions share the non-uniform FFT approximation with CLEAN. Both continually cycle between Visibility an image space. Compared to CLEAN, Compressed Sensing algorithms need more cycles to converge, which is one of the reasons why they have higher runtime costs. In this project, we postulated that one may reduce the runtime cost of Compressed Sensing reconstructions by replacing the non-uniform FFT approximation. 

We discussed three alternatives: Optimal Projection on a uniform Grid, using Spherical Harmonics and the direct Fourier Transform. With the latter approach, we created a proof-of-concept image reconstruction using  Coordinate Descent. We leveraged the starlet transform and created an algorithm which scales with the number of non-zero basis functions. We demonstrated super-resolution performance on simulated MeerKAT data and extrapolated the runtime costs on a real-world observation. Compared to CLEAN and the non-uniform FFT, our algorithm leads to higher runtime costs. As it is implemented in this project, the direct Fourier Transform is not a viable alternative on MeerKAT data. The other two alternatives both have drawbacks which need to be resolved before they become viable.

Projecting on a uniform grid is way of potentially reducing the problem to a more manageable size. In this architecture, the first step is to use the optimal projection on a uniform grid. Later processing steps like reconstruction can be done on the uniform grid. However, self-calibration requires a transformation from uniform sampled grid back to the original measurements. In this task, we cannot ignore the original Visibilities in later processing. Unless one finds a way to project the calibration problem on a uniform grid too, self-calibration destroys the main advantage of the approach.

Spherical Harmonics are a new way to represent the Radio Interferometric Measurement Equation. We analysed published research in this area\cite{carozzi2015imaging, mcewen2008simulating} and see potential for a new reconstruction architecture, which may reduce the overall runtime costs. However, the current formalism\cite{smirnov2011revisiting} is based on the plane wave. How a real-world imaging algorithm with spherical harmonics deals with calibration and ionosphere distortion may be fundamentally different to current CLEAN and Compressed Sensing reconstructions. At this point, it is unclear if or how many imaging problems have to be re-investigated by moving to Spherical Harmonics.





There is no clear alternative to the non-uniform FFT approximation. It seems runtime efficient compared to the alternatives. MeerKAT only increases its data volume. Future imaging algorithms have to move to distributed computing eventually. State-of-the-art CLEAN can distribute some steps of the reconstruction. But not completely, and limited. Research is focussed on getting the non-uniform FFT, CLEAN and Compressed Sensing approaches to distribute. It is an inherently iterative reconstruction algorithm.

Here, the direct Fourier Transform can become interesting. It does not need an iterative approximation scheme, and may prove to be useful for distributed computing. If we can get the runtime costs to be more competitive.

Bayesian statistics.
 
Still an open question, we do not have any break-through 








