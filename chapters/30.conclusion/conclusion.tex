\section{Compressed Sensing reconstructions in the MeerKAT era}
The MeerKAT interferometer poses the reconstruction problem on a large scale. Future expansions will lead to an even higher data volume.

 Compressed Sensing approaches exist, they use the major cycle, but they all are slower to compute than CLEAN, for which the major cycle was developed. 

It uses the non-uniform FFT to approximate the Fourier transformation of non-uniformly sampled Visibilities.

This project set out to explore different architectures for Compressed Sensing Reconstructions. A different way of getting from Visibilities to image. 

Three different possibilities, with calculating the non-uniform FFT once, using spherical harmonics, or calculating only columns of the Fourier transform matrix

For the former, we created a novel approach which does not need the major cycle. We need a transformation, where we can both estimate which components are likely to be non-zero, and which use few non-zero components to represent an image.

This project used starlets, which were developed for astronomical images. Their solution is not sparse enough to scale to MeerKAT problems.

This leaves two other ways, splitting the major cycle falls off if we consider self-calibration



Spherical harmonics are the last way,
Compressed Sensing algorithms have used it to gain another improvement in reconstruction quality. At this point, there is no Compressed Sensing algorithm which used the sphere for speed up.

here I see a possibility. The downside is it leads to a new measurement equation. It did not have as much time to formalize corrupting effects like antenna beam pattern, ionosphere and so on.

There is yet an



 
Still an open question








