\section{Compressed Sensing reconstructions in the MeerKAT era}
Compressed Sensing reconstructions in Radio Astronomy share the non-uniform FFT approximation with CLEAN. Both continually cycle between Visibility an image space. Compared to CLEAN, Compressed Sensing algorithms need more cycles to converge, which is one of the reasons why they have higher runtime costs. In this project, we postulated that one may reduce the runtime cost of Compressed Sensing reconstructions by replacing the non-uniform FFT approximation. 

We discussed three alternatives: Optimal Projection on a uniform Grid, using Spherical Harmonics and the direct Fourier Transform. 

With the latter approach, we created a proof-of-concept image reconstruction using  Coordinate Descent. We leveraged the starlet transform and created an algorithm which scales with the number of non-zero basis functions. We demonstrated super-resolution performance on simulated MeerKAT data and extrapolated the runtime costs on a real-world observation. Compared to CLEAN and the non-uniform FFT, our algorithm leads to higher runtime costs. As it is implemented in this project, the direct Fourier Transform is not a viable alternative on MeerKAT data.

Optimal Projection is another alternative to the non-uniform FFT and the direct Fourier Transform. It is way of reducing the problem to a more manageable size, since the total grid size is usually a fraction of the total measurements. In this architecture, the first step is to search for the optimal projection of Visibilities on a uniform grid. We then reconstruct the image with uniformly sampled Visibilities and do not need the original measurements. However, self-calibration requires a transformation from uniform sampled grid back to the original measurements. In this task, we cannot ignore the original Visibilities in later processing. Unless one finds a way to project the calibration problem on a uniform grid too, self-calibration destroys the main advantage of the approach.

The other alternative is using Spherical Harmonics instead of the Fourier Transform, which is a new way to represent the Radio Interferometric Measurement Equation. We analysed published research in this area\cite{carozzi2015imaging, mcewen2008simulating} and see potential for a new reconstruction architecture, which may reduce the overall runtime costs. However, the current formalism\cite{smirnov2011revisiting} is based on the plane wave. How a real-world imaging algorithm with spherical harmonics deals with calibration and ionosphere distortion may be fundamentally different to current CLEAN and Compressed Sensing reconstructions. At this point, it is unclear if or how many imaging problems have to be re-investigated by moving to Spherical Harmonics.

In this project we have not found a clear alternative to the non-uniform FFT. It leads to an iterative reconstruction algorithm with overall lower runtime costs. However, the iterative nature becomes troublesome for large scale distribution. New research has found ways to distribute parts of the algorithm\cite{offringa2014wsclean, pratley2018fast}, but how to distribute the reconstruction on a large scale is still an open problem.


The direct Fourier Transform does not need iterative approximation algorithms and lends itself to distributed computing.  


One advantage of the direct Fourier Transform is it simplifies distributed computing. The drawback of the direct Fourier Transform is that on MeerKAT data, it has an impractically large memory requirement or results in higher runtime costs. Our Coordinate Descent approach significantly reduces the memory requirement and runtime costs of the direct Fourier Transform. It is a step towards making the direct Fourier Transform more competitive, but our improvement alone is not enough. The memory requirement and runtime costs may be further reduced by leveraging the inherent redundancy in MeerKAT measurements. If this leads to a comparable runtime cost to the non-uniform FFT, the direct Fourier Transform becomes interesting for distributed computing.










