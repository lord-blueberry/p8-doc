\section{Wide Field of View Imaging} \label{radio}
So far the small Field of View inverse problem has been introduced where each antenna pair measures a Visibility of the sky brightness distribution. This leads to the small Field of View measurement equation \eqref{radio:eq:2dft}. It is identical to the two dimensional Fourier Transform. In practice the Fast Fourier Transform (FFT) is used, since it scales with $~n\:log(n)$ instead of $~n^2$ pixels.

\begin{equation}\label{radio:eq:2dft}
V(u, v) = \int\int x(l, m) e^{2 \pi i (ux+vy)} dl dm
\end{equation}

For wide Field of View imaging, two effects break the two dimensional Fourier Transform relationship: Non-coplanar Baselines and the celestial sphere which lead to the measurement equation \eqref{radio:eq:ftSphere}. Note that for small Field of View $1 - x^2 -y ^2 \ll 1$, and \eqref{radio:eq:ftSphere} reduces to the 2d measurement equation \eqref{radio:eq:2dft}.

\begin{equation}\label{radio:eq:ftSphere}
	V(u, v, w) = \int\int \frac{X(x, y)}{\sqrt{1 - x^2 - y ^2}} e^{2 \pi i (ux+vy+ w\sqrt{1 - x^2 - y ^2})}dx dy
\end{equation}

\begin{wrapfigure}{r}{0.5\textwidth}
	\centering
	\includegraphics[width=0.9\linewidth]{./chapters/03.radio/uvw.png}
	\caption{U V and W coordinate space}
	\label{radio:uvw}
	\vspace{-10pt}
\end{wrapfigure}

Non-coplanar Baselines lead to a third component $w$ for each Visibility. Figure \ref{radio:uvw} shows the the $u$ $v$ and $w$ coordinate system. $w$ is essentially the pointing direction of the instrument. The UV-Plane is the projection of the antennas on a plane perpendicular to the pointing direction. Which point in the UV-Plane get sampled and what $w$ component it has depends on the pointing direction. If the instrument points straight up, the UV-Plane is a tangent to earth's surface, and the $w$ term compensates for earth's surface curvature. If however the instrument points at the horizon, the projected UV-Plane gets squashed and $w$ compensates for antennas which lie far behind the UV-Plane. In essence, $w$ is a phase delay that corrects antenna positions in three dimensions. The wide Field of View measurement equation \eqref{radio:eq:ftSphere} would account for the $w$ phase delay, but it breaks the the two dimensional Fourier relationship and the FFT cannot be used. The W-Projection \cite{cornwell2008noncoplanar} algorithm approximates the effect of the $w$ term restores the two dimensional Fourier relationship.

\subsection{Interpretation of the W-Term and Spherical Harmonics}

A-Projection \cite{bhatnagar2008correcting}

\subsection{Separation of the W-term}
WSClean



%"The field of view of a telescope is limited by the primary beams of the antennas. To map a region of sky where the emission is at a scale larger than the angular width of the primary beams, mosaicing needs to be done. This is discussed in the second part of this lecture."

%Phase 

%source \cite{lfraSchool}

\subsection{Calibration}
%A lot of effects, weather, noise, antenna temperature, drift.

%Antenna based calibration, holds true for current interferometers but is not true for SKA. Possible switch to baseline based calibration.

